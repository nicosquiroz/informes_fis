% ****** Start of file aipsamp.tex ******
%
%   This file is part of the AIP files in the AIP distribution for REVTeX 4.
%   Version 4.1 of REVTeX, October 2009
%
%   Copyright (c) 2009 American Institute of Physics.
%
%   See the AIP README file for restrictions and more information.
%
% TeX'ing this file requires that you have AMS-LaTeX 2.0 installed
% as well as the rest of the prerequisites for REVTeX 4.1
%
% It also requires running BibTeX. The commands are as follows:
%
%  1)  latex  aipsamp
%  2)  bibtex aipsamp
%  3)  latex  aipsamp
%  4)  latex  aipsamp
%
% Use this file as a source of example code for your aip document.
% Use the file aiptemplate.tex as a template for your document.
\documentclass[%
 aip,
%jmp,%
%bmf,%
%sd,%
rsi,%
 amsmath,amssymb,
%preprint,%
 reprint,%
%author-year,%
%author-numerical,%
]{revtex4-1}

\usepackage{graphicx}% Include figure files
\usepackage{dcolumn}% Align table columns on decimal point
\usepackage{bm}% bold math
%\usepackage[mathlines]{lineno}% Enable numbering of text and display math
%\linenumbers\relax % Commence numbering lines


\begin{document}

\preprint{AIP/123-QED}

\title{An\'alisis num\'erico del sistema P\'endulo acoplado en el l\'imite de oscilaciones peque\~nas.}% Force line breaks with \\

\author{Nicol\'as Sep\'ulveda}
 \email{nicosepulveda4673@gmail.com}
 \affiliation{Pontificia Universidad Cat\'olica de Valpara\'iso%\\This line break forced with \textbackslash\textbackslash
}%



\date{\today}% It is always \today, today,
             %  but any date may be explicitly specified

\begin{abstract}
Se resuelven num\'ericamente las ecuaciones de movimiento de un p\'endulo acoplado, formado por dos p\'endulos simples cuyas masas son unidas por un resorte d\'ebil. Se revisan estados de oscilaci\'on asociados a los modos normales del sistema, a la superposici\'on de estos, y a la respuesta dado un t\'ermino disipativo y una fuerza peri\'odica aplicada en uno de los p\'endulos.

\end{abstract}

\maketitle












\section{Introducci\'on}



El sistema de p\'endulos acoplado consiste en dos p\'endulos simples, cuyas masas son unidas por un resorte, como se ve en la figura \ref{esquema}. El movimiento est\'a restringido a dos dimensiones, y la fuerza el\'astica que act\'ua en ambas es mucho menor que la fuerza de gravedad sobre ellas. La simpleza del sistema permite comprender c\'omo se comportan sistemas an\'alogos de acoplamiento tanto en escalas macro como microsc\'opicas. Para estudiarlo, se resolver\'an num\'ericamente las ecuaciones de movimiento mediante Python, bas\'andose en una soluci\'on para un sistema de masas acopladas\cite{sist_acoplado}, y se expondr\'an figuras que muestran los tipos de movimiento fundamentales que lo caracterizan, realizando una breve discusi\'on de cada uno de ellos. En primera instancia se simula el caso ideal, para obtener una representaci\'on visual de los batidos producidos en el sistema; luego se agregan t\'erminos de forzamiento y disipaci\'on.
\begin{figure}
    \centering
    {\includegraphics[width=\columnwidth]{figura1.pdf}
    \caption{\label{esquema} Diagrama del sistema. El \'angulo se mide con respecto a la vertical del p\'endulo $1$ y $2$ respectivamente. El resorte tiene constante el\'astica $k$. .}}
\end{figure}

\section{Ecuaciones de movimiento}
Luego de la simplificaci\'on $sin \theta \approx \theta$ para \'angulos peque\~nos, la ecuaci\'on de movimiento para los p\'endulos $1$ y $2$ respectivamente es:
\begin{eqnarray}
\ddot{\theta_1}+\frac{g}{L_1}\theta_1+\frac{k}{m_1}(\theta_1-\theta_2)=0
\label{pendulo1}
\\
\ddot{\theta_2}+\frac{g}{L_2}\theta_2+\frac{k}{m_2}(\theta_1-\theta_2)=0
\label{pendulo2},
\end{eqnarray}
en donde $\theta_1$ y $\theta_2$ son los \'angulos medidos para los p\'endulos de largo $L_1$ y $L_2$ y masa $m_1$ y $m_2$; $k$ la constante el\'astica del resorte, y $g$ la aceleraci\'on de gravedad.

En esta idealizaci\'on del sistema, y para p\'endulos id\'enticos, se pueden encontrar soluciones en las cuales la frecuencis de oscilaci\'on de ambos p\'endulos es la misma. Estos estados corresponden a los modos normales del sistema; uno dependiente de la frecuencia natural de los p\'endulos, y otro asociado a la fuerza ejercida por el resorte. Dada la ortonormalidad entre estos, cualquier movimiento del sistema puede ser descrito como una combinaci\'on lineal de ambos, y es en tales movimientos en donde se puede observar una transferencia de energ\'ia de un p\'endulo a otro, cuya proporci\'on depende de las condiciones iniciales. El intercambio entonces genera el fen\'omeno de batidos, pues las frecuencias naturales de cada modo son distintas. En un caso espec\'ifico a estudiar, en donde las amplitudes de oscilaci\'on son iguales, se da que las soluciones para ambos p\'endulos son:

\begin{eqnarray}
\theta_1=[2Acosw_bt]cosw_{prom}t\label{totalenergia1}\\ \theta_2=[2Asenw_bt]senw_{prom}r\label{totalenergia2},
\end{eqnarray}
en donde $A$ es la amplitud, $w_b$ la frecuencia del batido y $w_{prom}$ la frecuencia promedio entre ambos p\'endulos. Se hace notar que se cumple $w_b<<w_{prom}$\cite{batidos}.


Cuando en las ecuaciones de movimiento se agrega el t\'ermino de disipaci\'on dependiente de la velocidad angular $\gamma \dot{\theta}$, con la constante impuesta igual para ambos p\'endulos, se esperan soluciones moduladas por una exponencial atenuante, es decir, cuya amplitud disminuya con el tiempo: ecuaciones de oscilador amortiguado.

Adem\'as, agregando un t\'ermino de forzamiento peri\'odico en el primer p\'endulo, expresado por $Fcos(at)$, con F la amplitud y $a$ la frecuencia, se obtienen soluciones en donde es posible estudiar el fen\'omeno de resonancia del sistema.

\section{Simulaci\'on}
Usando el m\'odulo Scipy de Python, se han solucionado las ecuaciones de movimiento del sistema bajo diferentes par\'ametros y condiciones iniciales. Se escogieron valores espec\'ificos para simular los estados que se describen a continuaci\'on.

\subsection{Sistema ideal}
Bajo condiciones iniciales espec\'ificas, el sistema tiene comportamientos caracter\'isticos, como los modos normales, y la transferencia de energ\'ia de un oscilador a otro.
%-------------------SUBFIGURE
\begin{itemize}
    \item{\textbf{Modos.}} Como fue mencionado, existen dos modos normles para el sistema. Uno asociado a la frecuencia natural de los p\'endulos, y el otro a la elasticidad del resorte. El primero se ha simulado con la condici\'on inicial $\theta_1=\theta_2$, lo cual indica un movimiento id\'entico en ambos p\'endulos, si el resorte no est\'a actuando sobre las masas. El segundo, tiene como condiciones iniciales $\theta_1=-\theta_2$, lo que genera un movimiento contrario entre ambos p\'endulos, siendo atra\'idos entre s\'i por el resorte cuando las masas est\'an en su separaci\'on m\'axima. En las figuras \ref{f:modo1} y \ref{f:modo2} se observan ambas soluciones.
    
    
    \begin{figure}
    \subfigure{\includegraphics[width=\columnwidth]{Modo1.pdf}
    \caption{\label{f:modo1} Posici\'on angular de los dos p\'endulos en funci\'on del tiempo. Se observa que oscilan con misma fase y amplitud.}}
    \subfigure{\includegraphics[width=\columnwidth]{Modo2.pdf}
    \caption{\label{f:modo2}\ Posic\'on angular en funci\'on del tiempo. Ambos oscilan con la misma frecuencia y amplitud, pero desfasados en $\pi\, rad$. Con $k$ mayor, la frecuencia aumenta.}}
    
\end{figure}

    \item{\textbf{Superposici\'on de modos.}} Cuando el sistema no se encuentra oscilando en ninguna de sus frecuencias naturales, el movimiento se puede expresar como una suma de los modos, con respectivas amplitudes. En la figura \ref{totalenergia} se observa el movimiento descrito por las ecuaciones \ref{totalenergia1} y \ref{totalenergia2}, en el cual uno de los dos p\'endulos comienza en $\theta=0$. Este comienza a oscilar con un aumento de amplitud, mientras que el otro cede su energ\'ia y termina en oscilaciones peque\~nas. El ciclo se repite alternadamente, y se generan los patrones de \textit{batidos}. En la figura \ref{nototalenergia}, las condiciones iniciales son $\theta_1\neq\theta_2\neq0$, y se observa el mismo fen\'oneno de transferencia, pero no total, pues el segundo p\'endulo no inici\'o en su posici\'on de equilibrio.
    
    
    
    \begin{figure}
     \subfigure{
\includegraphics[width=\columnwidth]{superposicion1.pdf}
\caption{\label{totalenergia} Soluci\'on con condici\'on inicial $\theta_1=\pi/3;\,\theta_2=0$. La transferencia de energ\'ia en el batido es total.}}
\subfigure{\includegraphics[width=\columnwidth]{superposicion.pdf}
\caption{\label{nototalenergia} Soluci\'on con condici\'on inicial $\theta_1=\pi/3;\,\theta_2=\pi/6$. Las amplitudes iniciales se intercambian con una frecuencia mucho menor que la de las oscilaciones.}}
\end{figure}
\end{itemize}
\subsection{Sistema con disipaci\'on y forzamiento}

Ahora se introduce en las ecuaciones de movimiento un t\'ermino disipativo $\gamma \dot{\theta}$; y en la del p\'endulo $1$ uno de forzamiento $Fcos(at)$\cite{forzamiento}, con $F$ la amplitud y $a$ la frecuencia. La ecuaci\'on de este p\'endulo se transforma en
\begin{equation}
\ddot{\theta_1}+\gamma\dot{\theta}+\frac{g}{L_1}\theta_1+\frac{k}{m_1}(\theta_1-\theta_2)+Fcos(at)=0.
\end{equation}

En las figuras \ref{forzado1} y \ref{forzado2} se observan los decaimientos exponenciales asociados al t\'ermino disipativo, cuya magnitud en la simulaci\'on es arbitraria e igual para ambas, y escogida para observar el decaimiento dentro de la l\'inea de tiempo. En base a estas, se puede realizar una comparaci\'on cualitativa entre diferentes frecuencias de forzamiento. La magnitud de la primera es aproximadamente un tercio de la segunda. La principal diferencia se observa en la respuesta de las amplitudes de ambos p\'endulos. En el segundo caso se ha buscado mediante variaciones del par\'ametro $a$ la respuesta de mayor amplitud, con la intensi\'on de simular un movimiento resonante. Se observa en la figura c\'omo la amplitud del p\'endulo $2$ incrementa con la del $1$, en comparaci\'on al primer caso; en donde la amplitud del p\'endulo forzado es peque\~na y la del otro es aproximadamente nula.

  \begin{figure}
     \subfigure{
  \includegraphics[width=\columnwidth]{forzado1.pdf}
   \caption{\label{forzado1}Movimiento con forzamiento y disipaci\'on. Las condiciones iniciales son iguales al caso anterior, pero con $\gamma=0.5$, $F=1$ y $a=1$}.}
\subfigure{\includegraphics[width=\columnwidth]{forzado2.pdf}
\caption{\label{forzado2}En este caso, se ha cambiado la frecuencia de la fuerza externa a $a=3.2$, en donde se observa una mayor amplitud de los p\'endulos.}}
\end{figure}


\section{Conclusi\'on}
Se han estudiado, mediante simulaci\'ones num\'ericas en Python, las diferentes soluciones de movimiento para un oscilador acoplado, haciendo \'enfasis en los modos normales y batidos. Adem\'as se han agregado t\'erminos de disipaci\'on y forzamiento para analizar la evoluci\'on de un sistema mas cercano a la realidad. Las gr\'aficas de las posiciones de los p\'endulos en funci\'on del tiempo han permitido un an\'alisis visual de los comportamientos, dando una base para el estudio de sistemas m\'as complejos.









\begin{thebibliography}{10}

\bibitem{sist_acoplado}\url{https://scipy-cookbook.readthedocs.io/items/CoupledSpringMassSystem.html}
\bibitem{batidos}\url{https://physics.nyu.edu/~physlab/Classical%20and%20Quantum%20Wave%20Lab/cp.pdf}
\bibitem{forzamiento}Jarvis M.,{\it Waves \& Normal Modes}, 2016.

\url{https://www2.physics.ox.ac.uk/sites/default/files/2012-09-04/normalmodes_iandii_pdf_96820.pdf}
\end{thebibliography}
\end{document}
%
% ****** End of file aipsamp.tex ******